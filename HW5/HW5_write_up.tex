\documentclass[11pt,english]{article}

\usepackage[latin9]{inputenc}
\usepackage[letterpaper]{geometry}
\geometry{verbose,tmargin=1in,bmargin=1in,lmargin=1in,rmargin=1in}
\usepackage{babel}
\usepackage{amsmath}
\usepackage{amssymb}
\usepackage{capt-of}
\usepackage{graphicx}
\usepackage[usenames,dvipsnames]{color}
\usepackage{latexsym}
\usepackage{xspace}
\usepackage{pdflscape}
\usepackage[hyphens]{url}
\usepackage[colorlinks]{hyperref}
\usepackage{enumerate}
\usepackage{ifthen}
\usepackage{float}
\usepackage{array}
\usepackage{tikz}
\usetikzlibrary{shapes}
\usepackage{algorithm2e}

\newcommand{\rthree}{\mathbb{R}^3}
\title{MEAM 620 Advanced Robotics: Assignment 5\\
Due:  Wednesday April 8th}
 \author{Stephen Phillips}
\date{\today}

\newcommand{\vect}[2]{\begin{pmatrix} #1 \\ #2 \end{pmatrix}}

\begin{document}
\maketitle

\section*{Problem 1}
\subsection{Part a}
We simply implemented the continuous Kalman filter using Euler integration. The $C$ measurement
matrix is just the identity.

\begin{figure}[H]
  \centering
  \includegraphics[width=0.5\textwidth]{partajerk.eps}
  \caption{Jerk Estimates. The black line is the true $x$, the cyan is the estimated $x$, with the noise in the background}
\end{figure}

\begin{figure}[H]
  \centering
  \includegraphics[width=0.5\textwidth]{partaacc.eps}
  \caption{Acceleration Estimates. The black line is the true $x$, the cyan is the estimated $x$, with the noise in the background}
\end{figure}

\begin{figure}[H]
  \centering
  \includegraphics[width=0.5\textwidth]{partavel.eps}
  \caption{Velocity Estimates. The black line is the true $x$, the cyan is the estimated $x$, with the noise in the background}
\end{figure}

\begin{figure}[H]
  \centering
  \includegraphics[width=0.5\textwidth]{partapos.eps}
  \caption{Position Estimates. The black line is the true $x$, the cyan is the estimated $x$, with the noise in the background}
\end{figure}

\subsection{Part b}
Now we change the measurement matrix $C$ to be the row vector $[0,0,0,1]$. Note that the lower derivatives other
than position do poorly especially toward the end. Position does OK since we have measurements to filter it,
but unfortunately for the velocity and acceleration, we cannot do much better than integrate the inputs. So we
get worse estimates.

\begin{figure}[H]
  \centering
  \includegraphics[width=0.5\textwidth]{partbjerk.eps}
  \caption{Jerk Estimates. The black line is the true $x$, the cyan is the estimated $x$, with the noise in the background}
\end{figure}

\begin{figure}[H]
  \centering
  \includegraphics[width=0.5\textwidth]{partbacc.eps}
  \caption{Acceleration Estimates. The black line is the true $x$, the cyan is the estimated $x$, with the noise in the background}
\end{figure}

\begin{figure}[H]
  \centering
  \includegraphics[width=0.5\textwidth]{partbvel.eps}
  \caption{Velocity Estimates. The black line is the true $x$, the cyan is the estimated $x$, with the noise in the background}
\end{figure}

\begin{figure}[H]
  \centering
  \includegraphics[width=0.5\textwidth]{partbpos.eps}
  \caption{Position Estimates. The black line is the true $x$, the cyan is the estimated $x$, with the noise in the background}
\end{figure}


\subsection{Part c}
Now we change the measurement matrix $C$ to be the $3 \times 4$ matrix:
\[
C = \begin{pmatrix}
	0 & 1 & 0 & 0 \\
	0 & 0 & 1 & 0 \\
	0 & 0 & 0 & 1 \\
\end{pmatrix}
\]

This would give pretty good estimates for everything, even the position, since we have measurements for most of
our state, our estimate for the position can be well approximated by integrating the higher derivatives, namely
the velocity. So it would not diverge like the estimates in part b did. This shows how important it is to have
good measurements, otherwise we cannot give back the whole state. 


\end{document}

\documentclass[english]{article}
%%%%%%%%%%%%%%%%%%%%%%%%%%%%%%%%%%%%%%%%%%%%%%%%%%%%%%%%%%%%%%%%%%%%%%%%%%%%%%%%%%%%%%%%%%%%%%%%%%%%%%%%%%%%%%%%%%%%%%%%%%%%
\usepackage{latexsym,amsmath,amssymb,amsfonts,fullpage}


\begin{document}
	
\begin{center}
{\textbf{MEAM 620 Homework 2}} \\
Due: February 18, 2015, 11:59pm
\end{center}


\paragraph{1.} List two advantages and two disadvantages each for using (a) rotation matrices;(b) axis angle representation; (c) exponential coordinates; (d) Euler angles; and (e) quaternions to describe rotations? 
\[\textrm{ }\]
\begin{enumerate}
 \item[(a)] Rotation matrices 
    \begin{itemize}
     \item Advantages
     \begin{enumerate}
      \item[1.] Very easy to use in coordinates - just use matrix multiplication with point you want to rotate
      \item[2.] Covers all of $SO(3)$, since rotation matrices are how $SO(3)$ is defined. 
     \end{enumerate}
     \item Disadvantages
     \begin{enumerate}
      \item[1.] Not very compact - only need 3 numbers to represent $SO(3)$, but here we use 9
      \item[2.] Interpolation - interpolating between 2 rotation matrices is non-trival, and we often want to do this.
     \end{enumerate}
    \end{itemize}

 \item[(b)] Axis-Angle Representation 
    \begin{itemize}
     \item Advantages
     \begin{enumerate}
      \item[1.] Compact representation - just need 4 numbers to represent it (3 for axis, 1 for angle). 
      \item[2.] Covers all of $SO(3)$ - every matrix can be represented this way (similar to axis-angle) (with singularities as shown below)
     \end{enumerate}
     \item Disadvantages
     \begin{enumerate}
      \item[1.] Many to one representation - the zero rotation has infinitely many representations,
                and even if you restrict the angle to $[0,pi]$, then you get $R(x,\pi) = R(-x,\pi)$
                (though the latter is an inherent problem in $SO(3)$). 
      \item[2.] More difficult to compute with - need to convert into rotation matrix before you can use
     \end{enumerate}
    \end{itemize}

 \item[(c)] Exponential Coordinates 
    \begin{itemize}
     \item Advantages
     \begin{enumerate}
      \item[1.] Compact representation - just need 3 numbers to represent it (3 for axis times the angle) (Can also represent as axis and angle). 
      \item[2.] Covers all of $SO(3)$ - every matrix can be represented this way (similar to axis-angle), and avoids singularities of axis-angle
     \end{enumerate}
     \item Disadvantages
     \begin{enumerate}
      \item[1.] More difficult to compute with - need to convert into rotation matrix before you can use, like axis-angle
      \item[2.] Many to one map - This still is a many to one representation of $SO(3)$, since we have the $(x,\pi)$ and $(-x,\pi)$
                axis angle singularity.
     \end{enumerate}
    \end{itemize}

 \item[(d)] Euler angles 
    \begin{itemize}
     \item Advantages
     \begin{enumerate}
      \item[1.] Compact representation - just need 3 numbers to represent it, one for each angle.  
      \item[2.] Easy to compute with - just need to compute the 3 rotation matrices to do it (which is comparitively easy)
     \end{enumerate}
     \item Disadvantages
     \begin{enumerate}
      \item[1.] Singularities - Gimbal lock is a big problem if you are dealing with large rotations
      \item[2.] 
     \end{enumerate}
    \end{itemize}

 \item[(e)] Quaternions 
    \begin{itemize}
     \item Advantages
     \begin{enumerate}
      \item[1.] Compact representation - just need 4 numbers to represent it, one for each angle.
      \item[2.] Easy to compute with - It is very fast to compute a rotation with a quaternion - just use $qpq^{-1}$. Also easy to compose quaternions
     \end{enumerate}
     \item Disadvantages
     \begin{enumerate}
      \item[1.] Renormalization - quaternions MUST be of unit length, so you have to renormalize every time you combine quaternions
      \item[2.] 
     \end{enumerate}
    \end{itemize}

\end{enumerate}



\paragraph{2.}  
Consider the problem of fitting a smooth curve to the following waypoints in 2D: 
\begin{align*}
t_0 &= 0, (x_0, y_0) = (-1, 0) \\
t_1 &= 5, (x_1, y_1) = (0, 2) \\
t_2 &= 6, (x_2, y_2) = (1, 0) \\
t_0 &= 0, (\dot{x}_0, \dot{y}_0) = (-1, -5) 
\end{align*}

Note that the first three constraints are position constraints, while the last is a velocity constraint. Any other necessary derivative constraints at $t_0 = 0$ and $t_2 = 6$ should be set to $(0, 0)$. To minimize the functional:
\begin{align}
\int_{t = 0}^T \| x^{(n)} \|^2 dt,
\end{align}

\noindent the endpoints need to be constrained in position, velocity, and up to and including the $(n-1)$st derivative. All derivatives (velocity, acceleration, etc.) at $t_1 = 5$ should be left unspecified, and you will need to add the appropriate number of continuity constraints at that point. Also, note you will need to find $x(t)$ and $y(t)$.

a. A minimum acceleration trajectory can be constructed by fitting a cubic spline. Construct this trajectory for the waypoint constraints above. Explicitly write down the solution you find (ie. write down $x(t) = c_0 + c_1 t + c_2 t^2 ...$, where you fill in $c_0, c_1, c_2, ...$ with the coefficient values you found) and create a plot illustrating each trajectory and the waypoints. Include your Matlab code with your submission. 

b. What is the minimum order polynomial you need to construct a minimum jerk trajectory? Construct this trajectory, write down your solution, and create a plot as in part a. 

c. What is the minimum order polynomial you need to construct a minimum snap trajectory? Construct this trajectory, write down your solution, and create a plot. 

d. Name one advantage and one disadvantage of choosing a lower order polynomial over a higher one.  




\paragraph{3.}

Calculate the angular velocities $\omega^s$ and $\omega^b$ for the rotation:
\begin{align}
R = e^{\hat{\omega}_1 t} e^{\hat{\omega}_2 t}
\end{align}



\paragraph{4.} {
What skew-symmetric matrix $\hat{\omega} \in so(3)$ corresponds to the rotation
\begin{equation*}
R= \begin{bmatrix}
   -0.3038 &  -0.6313 & -0.7135\\
   -0.9332  &  0.3481   & 0.0893\\
    0.1920   & 0.6930 &  -0.6949\\
\end{bmatrix}
\end{equation*}


Is it unique?

%\bibliography{ref}
%\bibliographystyle{plain}

\end{document}

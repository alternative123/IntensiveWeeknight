\documentclass[11pt,english]{article}

%\usepackage[latin9]{inputenc}
\usepackage[letterpaper]{geometry}
\geometry{verbose,tmargin=1in,bmargin=1in,lmargin=1in,rmargin=1in}
\usepackage{babel}
\usepackage{amsmath}
\usepackage{amssymb}
\usepackage{graphicx}

\newcommand{\rthree}{\mathbb{R}^3}
\title{MEAM 620 Advanced Robotics: Homework 3\\
Due: Wednesday, March 4, 2015,  11:59am }
 \author{}
 
\date{}

\begin{document}
\maketitle
\begin{enumerate}
\item
(20pts) We have defined an edge point as a point where the gradient magnitude of the image $\| \nabla I \| $ reaches a local maximum along the gradient direction. This means that the derivative of $\| \nabla I \| $ along the gradient direction $\frac{\nabla I}{\| \nabla I \|}$ has a zero crossing. Compute 
\[
\nabla_\eta \| \nabla I \|
\qquad \mbox{where} \qquad  \eta = \frac{\nabla I}{\| \nabla I \|} .
\]
You have to know (look up) how to differentiate the magnitude of a vector $\| v \|$ with respect to the vector $v$. 

First we compute the gradient of the function $\|\nabla I\|$. We here use the notation that $\frac{\partial I}{\partial x} = I_x$ and
similarly for $I_y$. Using this we get:
\begin{align*}
    \nabla ||\nabla I|| 
    =&\; \left(\frac{\partial \sqrt{I_x^2 + I_y^2 }}{\partial x}, \frac{\partial \sqrt{I_x^2 + I_y^2}}{\partial y} \right)^\top \\
    =&\; \left(\frac{2I_x I_{xx}+2I_y I_{xy}}{2\sqrt{I_x^2+I_y^2}},\frac{2I_x I_{xy}+2I_y I_{yy}}{2\sqrt{I_x^2+I_y^2}}\right)^\top \\
    =&\; \frac{1}{\sqrt{I_x^2+I_y^2}} \left(I_x I_{xx}+I_y I_{xy},I_x I_{xy}+I_y I_{yy}\right)^\top \\
    =&\; \frac{1}{\|\nabla I\|} 
            \begin{bmatrix}
                I_{xx} & I_{xy} \\
                I_{xy} & I_{yy}
            \end{bmatrix} \nabla I \\
    =&\; \frac{1}{\|\nabla I\|} \left( \nabla^2 I\right) \nabla I
\end{align*}

So taking the directional derivative of $\frac{\nabla I}{\|\nabla I\|}$ we get:
\[ \nabla ||\nabla I|| \cdot \frac{\nabla I}{\|\nabla I\|} = \frac{1}{\|\nabla I\|^2} \nabla I^\top \left( \nabla^2 I\right) \nabla I \]

\item
(80pts)
Let $M$ be the autocorrelation matrix of a corner detector 
\[
M = \sum_{(x,y)\in \mathcal{N}(x_0,y_0)}  \begin{pmatrix}
I_x(x,y)^2 & I_x(x,y) I_y(x,y) \\ I_x(x,y) I_y(x,y) & I_y(x,y)^2\end{pmatrix}  .
\]

I will refer to this as the following:
\[
    M = \sum_{\vec{x}} \nabla I(\vec{x}) \nabla I(\vec{x})^\top 
\]

a. What will happen to the trace of the matrix if the image will be dilated $I'(x,y) = I(x/2,y/2)$.
Assume that $I_x,I_y$ are the image derivatives directly (without any Gaussian convolution) and that the neighborhood of summation is double the original size. 
\\
b. What will happen to the trace of the matrix if the image will be rotated by 45$\deg$ ?
\\
We look at rotated image as $I'(\vec{x}) = I(R\vec{x})$. Therefore by the chain rule
$\nabla I'(\vec{x}) = R \nabla I(R\vec{x})$. So now with this, we can recompute $M$:
\begin{align*}
    M'
    =&\; \sum_{\vec{x}} \nabla I'(\vec{x}) \nabla I'(\vec{x})^\top \\
    =&\; \sum_{\vec{x}} R\nabla I(R\vec{x}) \nabla I(R\vec{x})^\top R^\top \\ 
    =&\; R \left( \sum_{\vec{x}} \nabla I(R\vec{x}) \nabla I(R\vec{x})^\top \right) R^\top \\
    =&\; R M R^\top \\
\end{align*}

So now our task is to compute $\mathbf{tr}(R M R^\top)$. It can be shown that $\mathbf{tr}(AB) = \mathbf{tr}(BA)$:
\[ \mathbf{tr}(AB) = \sum_i (AB)_{ii} = \sum_i \sum_j A_{ij} B_{ji}
= \sum_j \sum_i B_{ji} A_{ij} = \sum_j (BA)_{jj} = \mathbf{tr}(BA)  \]

Therefore:
\[ \mathbf{tr}(RMR^\top) = \mathbf{tr}(R(MR^\top)) = \mathbf{tr}((MR^\top)R) = \mathbf{tr}(M) \]

So the trace is the same as the original, $\mathbf{tr}(M)$

c. Compute the eigenvalues of the matrix if the neighborhood contains only one straight edge at 45 degrees orientation:
\[
I(x,y) = \left\{
	\begin{array}{ll}
		1  & \mbox{if } x+y \geq 0 \\
		0 & \mbox{if } x+y < 0
	\end{array}
\right.
\]

To compute the eigenvalues of $M$, we need to know $\partial I / \partial x$ and $\partial I / \partial y$.
We define in class for discrete variables the derivative as $\frac{1}{2} I(x-1,y) - \frac{1}{2} I(x+1,y)$,
so we have:
\[ \frac{\partial I}{\partial x} = \left\{
    \begin{array}{ll}
       -1 & \textrm{if } x + y = 0 \\
        0 & \textrm{otherwise}
    \end{array} \right.
\]
\[ \frac{\partial I}{\partial y} = \left\{
    \begin{array}{ll}
       -1 & \textrm{if } x + y = 0 \\
        0 & \textrm{otherwise}
    \end{array} \right.
\]

This means that only the $(x,y)$ pairs of the form $(k,k)$ for some $k$ have nonzero derivative. Let's say
there are $n$ of them. We compute the
$M$ matrix:
\begin{align*}
M 
=&\; \sum_{\vec{x}} \nabla I(\vec{x}) \nabla I(\vec{x})^\top \\
=&\; \sum_{k} [-1 \; -1 ]^\top [ -1 \; -1 ]  \\
=&\; \sum_{k}
        \begin{bmatrix}
           1 & 1 \\
           1 & 1 
        \end{bmatrix} \\
=&\; n  \begin{bmatrix}
           1 & 1 \\
           1 & 1 
        \end{bmatrix} \\
\end{align*}

This clearly has eigen values $2n$ and $0$.
\[
n \begin{bmatrix}
           1 & 1 \\
           1 & 1 
   \end{bmatrix} 
 \begin{bmatrix}
           1  \\
           1  
   \end{bmatrix} =
 2n \begin{bmatrix}
           1  \\
           1  
   \end{bmatrix} 
\]

\[
n \begin{bmatrix}
           1 & 1 \\
           1 & 1 
   \end{bmatrix} 
 \begin{bmatrix}
           1  \\
          -1  
   \end{bmatrix} =
   \begin{bmatrix}
           0  \\
           0  
   \end{bmatrix} 
\]

This represents the apature problem: We can only see change in one direction. If we move the window accross the
$[ 1 \; -1 ]^\top$ direction, then we will see no change in the trace of $M$. 

d. In this last question we want to see whether the big red rectangle is a better Harris corner than the small one. 
% We will assume that the image reads
% \begin{figure}[htb]
% \centerline{\includegraphics[width=1in]{whiteRed.png}}
% \end{figure}
\[
I(x,y) = \left\{
	\begin{array}{ll}
		1  & \mbox{if } x^2+y^2  \leq r\\
		0 & \mbox{if } \mbox{otherwise}
	\end{array}
\right.
\]
yielding a  gradient in the direction of the radius $\nabla i = (\cos\theta,\sin\theta)$.
The large rectangle extends for $\theta=0..\frac{\pi}{4}$ while the small rectangle extends for  $\theta=0..\frac{\pi}{8}$. Compute the autocorrelation matrix in both cases by replacing the sum with an integral, i.e., compute $\int\int \frac{\partial{I}}{\partial{x}}\,\,dxdy$, etc. Compute in both cases the trace and the determinant. Which of the rectangle interiors has more ``cornerness'' ?


Since the gradient is zero everywhere except the circle, the $M$ matrix becomes:
\begin{align*}
    M =&\; \int_{\theta}[\cos\theta\;\sin\theta]^\top[\cos\theta\;\sin\theta] d\theta \\
    =&\; \int_{\theta}
        \begin{bmatrix}
            \cos^2 \theta & \sin\theta \cos\theta \\
            \sin\theta \cos\theta & \sin^2 \theta
        \end{bmatrix} d\theta \\
    =&\; \int_{\theta}
        \begin{bmatrix}
            \frac{1}{2} (1 + \cos 2\theta) & \frac{1}{2}\sin2\theta \\
            \frac{1}{2}\sin2\theta & \frac{1}{2} (1 - \cos 2\theta)
        \end{bmatrix} d\theta \\
    =&\; \frac{1}{2} \int_{\theta} \left(
        \begin{bmatrix}
            1 & 0 \\
            0 & 1
        \end{bmatrix} + 
        \begin{bmatrix}
            \cos 2\theta & \sin 2\theta \\
            \sin 2\theta & -\cos 2\theta
        \end{bmatrix} \right) d\theta \\
    =&\; \frac{1}{2} \int_{\theta=0}^{\theta_{max}}
        \begin{bmatrix}
            1 & 0 \\
            0 & 1
        \end{bmatrix} d\theta + \frac{1}{2} \int_{\theta}
        \begin{bmatrix}
            \cos 2\theta & \sin 2\theta \\
            \sin 2\theta & -\cos 2\theta
        \end{bmatrix} d\theta \\ 
    =&\; \frac{1}{2} \int_{\theta=0}^{\theta_{max}} 
        \begin{bmatrix}
            1 & 0 \\
            0 & 1
        \end{bmatrix} d\theta + \frac{1}{2} \int_{\theta=0}^{\theta_{max}}
        \begin{bmatrix}
            \cos 2\theta & \sin 2\theta \\
            \sin 2\theta & -\cos 2\theta
        \end{bmatrix} d\theta \\ 
    =&\; \frac{\theta_{max}}{2} 
        \begin{bmatrix}
            1 & 0 \\
            0 & 1
        \end{bmatrix} + \frac{1}{4} 
        \begin{bmatrix}
            \sin 2\theta_{max} - \sin 2 \cdot 0 & -\cos 2\theta_{max} + \cos 2\cdot 0 \\
            -\cos 2\theta_{max} + \cos 2\cdot 0 & -\sin 2\theta_{max} + \sin 2 \cdot 0
        \end{bmatrix} \\ 
    =&\; \frac{\theta_{max}}{2} 
        \begin{bmatrix}
            1 & 0 \\
            0 & 1
        \end{bmatrix} + \frac{1}{4} 
        \begin{bmatrix}
            0 & 1 \\
            1 & 0
        \end{bmatrix} + \frac{1}{4} 
        \begin{bmatrix}
            \sin 2\theta_{max}  & -\cos 2\theta_{max}  \\
            -\cos 2\theta_{max} & -\sin 2\theta_{max} 
        \end{bmatrix}  
\end{align*}

Now we just plug in the values for $\theta_{max}$. As the analytic numbers get confusing, we
will use numerical values rather than analytic values. We have for the larger of the two, 
$\theta_{max} = \pi/2$.

\[
M_1 = \frac{\pi}{4} 
\begin{bmatrix}
    1 & 0 \\
    0 & 1
\end{bmatrix} + \frac{1}{4} 
\begin{bmatrix}
    0 & 1 \\
    1 & 0
\end{bmatrix} + \frac{1}{4} 
\begin{bmatrix}
    \sin (\pi)  & -\cos (\pi) \\
    -\cos (\pi) & -\sin (\pi) 
\end{bmatrix}
= \begin{bmatrix}
    0.7854 & 0.5000 \\
    0.5000 & 0.7854
\end{bmatrix} \]

For the smaller one, we have $\theta_{max} = \pi/4$. 
\[
M_2 = \frac{\pi}{16} 
\begin{bmatrix}
    1 & 0 \\
    0 & 1
\end{bmatrix} + \frac{1}{4} 
\begin{bmatrix}
    0 & 1 \\
    1 & 0
\end{bmatrix} + \frac{1}{4} 
\begin{bmatrix}
    \sin (\pi/2)  & -\cos (\pi/2) \\
    -\cos (\pi/2) & -\sin (\pi/2) 
\end{bmatrix}
= \begin{bmatrix}
    0.6427 & 0.2500 \\
    0.2500 & 0.6427
\end{bmatrix} \]

The 'cornerness' of these matrices is determined by $c(A) = \det(A) - 0.06 \cdot \mathbf{tr}(A)^2$. Computing it on
each of these two matrices we get

\[c(M_1) = 0.2188\]
\[ c(M_2) = -0.0078 \]

Clearly $M_1$, the one
corresponding to $\theta_{max} = \pi/2$, is the better corner. 



\end{enumerate}



\end{document}
